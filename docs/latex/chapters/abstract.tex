\begin{abstract}

% Background: Describe the problem space you are working on, why it is important for the community, and why there is a need for this research.  
Bipedal robots collect data by internal sensors, and it is common to get basic data about the surroundings of the robot environment.
% Objective: State your research project objective, research questions, or hypotheses.
This paper is focused on problem of using the environmental sensor data in order to predict contact interaction scenario of a bipedal robot.
% Methods: Introduce the study design and methods you used for this work. 
Prediction of contact interaction scenario has been proposed. Our methodology utilizes a architecture of fully connected neural networks to utilize environmental data for predicting contact interaction scenario.
% Results: Briefly state what you accomplished or found, especially as it pertains to the objective stated earlier. 
As result, Our model can predict predicting contact interaction scenario with
 {\color{red}XX
 \footnote{Draft note: \color{red}the red marked texts will be finalized after actual results}
% Implications: Identify 1–2 implications or contributions of this work.
 }. The utilization of this method can enhance the performance of contact interaction scenario improvisation, such as slipping or take off from the supporting surface for bipedal robots.




\end{abstract}